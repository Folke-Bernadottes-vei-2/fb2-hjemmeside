% !TEX TS-program = LuaLaTeX
\documentclass[12pt]{article}

\usepackage[a4paper,margin=1in]{geometry}
\usepackage{enumitem}
\usepackage{hyperref}
\usepackage{libertine}
%\usepackage[norsk]{babel}
\usepackage[none]{hyphenat}

\title{\LARGE\bf Innkalling til sameiemøte}
\author{Ekstraorinært sameiemøte i Sameiet Folke Bernadottes vei 2}
\date{}

\setlength{\parindent}{0pt}
\setlength{\parskip}{1ex}
\setlength{\itemsep}{0.2ex}

\begin{document}

\begin{centering}
  {\Large\bf Referat fra ekstraordinært sameiemøte 4.\ juni 2024}
\end{centering}

\bigskip

Den 4.\ juni 2024 ble det avholdt et ekstraordinært sameiemøte for Sameiet Folke Bernadottes vei 2
på Zoom. Innkallingen finnes \href{http://fb2.no/arsmoter/2024-06-sameiemote/}{på hjemmesiden
  fb2.no}, der sakene er presentert.

\subsection*{Sak 1 - Konstituering}

{\bf 1a. Registrering av deltakere}

Ved møtets start var det registrert 5 fullmakter til styret der sameiere hadde gitt instruksjoner
om hvordan det skulle stemmes (også kjent som forhåndsstemmer). Norges Handicapforbund's
representant stilte med 10 stemmer og det var ytterligere ni stemmer til stede. Etter at sak 1c var
gjennomført, kom en siste sameier inn i møtet. Med dette ble det totalt {\bf 25 registrerte
  stemmer}.\smallskip

{\bf 1b. Valg av møteleder}

Styreleder Jenny Marie Myhre ble foreslått som møteleder uten motkandidater, og dermed valgt til
møteleder.\smallskip

{\bf 1c. Valg av protokollfører}

Varamedlem Pål Hermunn Johansen ble foreslått som protokollfører uten motkandidater, og ble dermed
valgt som protokollfører.\smallskip

{\bf 1d. Valg av to sameiere til å underskrive protokollen}

Sameierne Stein Laeskogen og Anders Folkestad ble valgt til å underskrive protokollen.

{\bf 1e. Spørsmål om møtet er lovlig innkalt}

Ingen av de oppmøtte hadde invendinger mot inkallingens lovlighet.

\subsection*{Sak 2 - Delfinansiering av brannvarsling i tre leiligheter}

En sameier mente det var feil at saken kom opp i et ekstraordinært sameiemøte, og at saken burde
avvises. Dette ble fremmet som et forslag for avstemming. I denne avstemningen ble det avgitt {\bf
  3 stemmer for} å avvisse sak 2, mens det ble avlagt {\bf 6 stemmer mot} forslaget. Det var {\bf
  16 blanke stemmer}. Dermed kunne saken fortsette.

For å presisere forslaget fra innkallingen, ønsket en sameier å vise til at sameiets utgifter skal
være begrenset til 6000 kroner. Derfor ble dette presisert i en ny formulering av det samme
forslaget, som følger

{\rule{\textwidth}{0.3pt}}

{\em Forslag til vedtak:}

Dersom tilbudet fra Firesafe presentert i sak 4j fra det ordinære
sameiemøtet 2. april i år aksepteres av alle de tre eierne av de aktuelle leilighetene, skal
Sameiet betale en tredjedel, men maksimalt 6000 kroner totalt for alle de tre leilighetene. Det
overlates til regnskapsfører å føre utgiften på en rimelig post i regnskapet.

{\rule{\textwidth}{0.3pt}}

Forslaget ble {\bf vedtatt} med {\bf 23 stemmer for} og {\bf 2 stemmer mot}.

\subsection*{Sak 3 - Godtgjørelse til styret}

Forslagsstiller la frem forslaget slik det var presentert i innkallingen.

En sameier stilte spørsmål ved om det var riktig å fastsette godtgjørelse til det midlertidige
styret og det nye styret før den ordinære generalforsamlingen neste år.

Medlemmer fra det midlertidige styret bemerket at det har vært mange små og store saker som har
kommet inn til styret i månedene som har vært, og at det tar mye tid og at uten en økning i
honoraret vil bety at det blir enda vanskeligere å rekruttere medlemmer til styret.

Under debatten ble det bemerket at det ikke nødvendigvis er en sammenheng mellom hvordan et styre
arbeider, og hvilket vederlag som utbetales.

Forslaget fra innkallingen ble så tatt til avstemming. For ordens skyld gjengis forslaget her:

{\rule{\textwidth}{0.3pt}}

{\em Forslag til vedtak:}

Det midlertidige styret som har arbeidet siden det ble insatt i det ordinære sameiermøtet dette år,
skal motta 20 000 kroner, som skal fordeles på vanlig måte.

Deretter skal det nye styret godtgjøres med 150 000 kroner for sitt arbeid som avsluttes i april
2025.

{\rule{\textwidth}{0.3pt}}

Forslaget ble {\bf vedtatt} ved avstemming, med {\bf 23 stemmer for}, {\bf ingen stemmer mot}
forslaget og {\bf 2 blanke stemmer}.

\subsection*{Sak 4 - Valg av nytt styre}

Forslag til nytt styre ble presentert, og deretter {\bf valgt ved akklamasjon}. Det betyr at det
nye styret er som følger:

\begin{tabular}{@{\hspace{4em}}ll}
  {\bf Styrets leder:} & Knut Noremsaune
  \vspace{1ex}\\
  {\bf Styremedlemmer:} & Jenny Marie Myhre \\
  & Andrii Bezkorovainyi
  \vspace{1ex}\\
  {\bf Vararepresentanter:} & Pål Hermunn Johansen\\
  & Elisabeth Skoglund
\end{tabular}\medskip

\vspace{3cm}

\begin{center}
\bgroup
\def\arraystretch{1.5}
\begin{tabular}{ccc}\cline{1-1}\cline{3-3}
  \hspace{2em}Stein Laerskogen\hspace{2em} & \hspace{7em} & \hspace{2em}Anders Folkestad\hspace{2em}
\end{tabular}
\egroup
\end{center}

\end{document}
