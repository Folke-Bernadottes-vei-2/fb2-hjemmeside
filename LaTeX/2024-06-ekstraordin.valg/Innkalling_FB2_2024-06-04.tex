% !TEX TS-program = LuaLaTeX
\documentclass[12pt]{article}

\usepackage[a4paper,margin=1in]{geometry}
\usepackage{enumitem}
\usepackage{hyperref}
\usepackage{libertine}
%\usepackage[norsk]{babel}
\usepackage[none]{hyphenat}

\title{\LARGE\bf Innkalling til sameiemøte}
\author{Ekstraorinært sameiemøte i Sameiet Folke Bernadottes vei 2}
\date{}

\setlength{\parindent}{0pt}
\setlength{\parskip}{1ex}
\setlength{\itemsep}{0.2ex}

\begin{document}

\begin{centering}
  {\LARGE\bf Innkalling til sameiemøte}

  \medskip

  {\large Ekstraordinært sameiemøte i Sameiet Folke Bernadottes vei 2}

\bigskip

\begin{tabular}{ll}
  {\bf Dato og tid:} & Tirsdag 4.\ juni 2024, klokken 18:00 \\
  & Møtet åpner for å ta imot møtedeltakere klokken 17:45
  \vspace{1ex}\\
  {\bf Sted:} & \href{https://us05web.zoom.us/j/86297034426?pwd=ffDzvL837fV5aNaDnQprRRzVqm4z8Q.1}{Zoom-møte}\\
  & Møte-ID: 862 9703 4426 \\
  & Passord: 5Y5k55
\end{tabular}
\end{centering}

\bigskip

Neste side inneholder informasjon om møte-plattformen Zoom og hvordan man kan koble seg til.\bigskip

{\bf Til behandling foreligger:}
\begin{enumerate}[topsep=3pt,itemsep=3pt,partopsep=3pt, parsep=3pt]
\item[1.] Konstituering
  \begin{enumerate}[topsep=3pt,itemsep=3pt,partopsep=3pt, parsep=3pt]
    \item[a.] Registrering av deltakere
    \item[b.] Valg av møteleder
    \item[c.] Valg av protokollfører
    \item[d.] Valg av to sameiere til å underskrive protokollen
    \item[e.] Spørsmål om møtet er lovlig innkalt
  \end{enumerate}
\item[2.] Delfinansiering av brannvarsling i tre leiligheter
\item[3.] Godtgjørelse til styret
\item[4.] Valg av nytt styre
\end{enumerate}

% \pagebreak
\section*{Angående digitalt avholdt møte}

\raggedright
\sloppy

Digitale årsmøter i sameier - og generalforsamlinger i borettslag - har blitt lovlig på permanent
basis. Styret ønsker at flest mulig skal kunne delta, og loven krever at det skal legges til rette
for at alle skal kunne delta. Styret har derfor valgt en møte-plattform som har god støtte av alle
de vanligste systemene for mobiltelefoner, nettbrett og datamaskiner. Zoom har apper/programmer for
Android, iOS, macOS, Linux og Windows, og det vil være mulig å bruke telefonen til å ringe inn og
delta (uten video).

For de som ikke kan delta på egne elektroniske enheter, vil det være mulig å gi fullmakt til en
annen sameier til å stemme på dems vegne. En slik fullmakt må meldes til styret, på ark,
tekstmelding eller e-post. Vi ønsker å ha slike fullmakter i hende dagen før møtet (eller
tidligere), slik at styret kan ha en best mulig oversikt når møtet starter.

En fullmakt til styret kan inkludere en forhåndsuttalelse og en instruks om hvordan det skal
stemmes i de ulike sakene. Forhåndsuttalelsene vil bli presentert på møtet, og kan også sendes på
forhånd til alle sameierne, hvis fullmaktsgiver ber om det.

Den beste måten å delta i møtet på er å følge lenken som ble delt sammen med e-posten med denne
innkallingen. Følg instruksjonene som dukker opp på skjermen for å komme inn i møtet.  Det vil være
mulig å komme inn i møterommet fra og med 15 minutter før møtet starter.

Vi håper at flest mulig vil delta i møtet via Zoom-applikasjonen, men ta kontakt med styret dersom
du vil delta via telefon, slik at vi kan teste det på forhånd. Vi ønsker ikke at eventuelle tekniske
problemer skal forhindre noen fra å delta i møtet.

% \pagebreak
\section*{2. Delfinansiering av brannvarsling i tre leiligheter}

I det ordinære sameiermøtet den 2.\ april dette år, ble det i sak 4j foreslått at Sameiet skulle
finansiere installasjon og tilkobling av nye bransvarslere. Bakgrunnen er at tidligere seksjon 7,
vanligvis omtalt som leilighet 23459, ble reseksjonert til fire små leiligheter, og at tre av disse
ikke har brannvarslere tilkoblet til det felles brannvasrslingssystemet.

Tilbudet fra Firesafe som ble presentert i det ordinære sameiemøtet står ved lag, og Styret
foreslår at Sameiet betaler en tredjedel av regningen, det vil si 6000 kroner, dersom eierne av de
tre leilighetene godtar tilbudet fra Firesafe. Dette vil bidra til å øke den generelle
brannsikkerheten i Sameiet samtidig som eierne av de aktuelle leilighetene tar brorparten av
kostnadene.

{\rule{\textwidth}{0.3pt}}

{\em Forslag til vedtak:}

Dersom tilbudet fra Firesafe presentert i sak 4j fra det ordinære sameiemøtet 2.\ april i år
aksepteres av eierne av de aktuelle leilighetene, skal Sameiet betale en tredjedel, som svarer til
6000 kroner. Det overlates til regnskapsfører å føre utgiften på en rimelig post i regnskapet.

{\rule{\textwidth}{0.3pt}}

\section*{3. Godtgjørelse til styret}

{\it Pål Hermunn Johansen, varamedlem og tidligere styreleder, skriver:}

Saken {\it Godtgjørelse til styret} var tidligere alltid en del av det ordinære sameiermøtet, men
saken har falt ut de siste årene. Jeg mener at det er riktig å gjeninnføre denne saken som en fast
sak, og at den også skal behandles i dette møtet.

Som vanlig handler saken om hva styret som innsettes nå, og avslutter sitt arbeid i april 2025,
skal få som godtgjørelse. I tillegg er det naturlig å bestemme hva det midlertidige styret skal få
utbetalt for arbeidet det har utført til nå.

Summen av disse beløpene vil finnes på budsjettet for 2025.

Siden det har vært en viss økning i konsumprisindeksen og lønningsnivået, og fordi det ordinære
sameiemøtet la mange oppgaver på Styret, ønsker jeg å justere beløpet opp noe, og har formulert det
i følgende forslag.

{\rule{\textwidth}{0.3pt}}

{\em Forslag til vedtak:}

Det midlertidige styret som har arbeidet siden det ble insatt i det ordinære sameiermøtet dette år,
skal motta 20 000 kroner, som skal fordeles på vanlig måte.

Deretter skal det nye styret godtgjøres med 150 000 kroner for sitt arbeid som avsluttes i april
2025.

{\rule{\textwidth}{0.3pt}}

Merk styret som avsluttet sitt arbeid i år, sammen har hatt en godtgjørelse på 150 000 kroner.

\section*{4. Valg av nytt styre}

I det ordinære sameiemøtet 2.\ april 2024 ble det ikke valgt et nytt permanent styre, men et
midlertidig interimstyre, bestående av følgende medlemmer:\bigskip

\begin{tabular}{@{\hspace{4em}}ll}
  {\bf Styrets leder:} & Jenny Marie Myhre
  \vspace{1ex}\\
  {\bf Styremedlemmer:} & Anders Folkestad\\
  & Knut Noremsaune
  \vspace{1ex}\\
  {\bf Vararepresentanter:} & Pål Hermunn Johansen\\
  & Elisabeth Skoglund
\end{tabular}\medskip

Den midlertidige styrelederen, Jenny Marie Myhre, er ikke villig til å være permanent styreleder,
men er villig til å være styremedlem. Knut Noremsaune har sagt seg villig til å tre inn som
styreleder og de øvrige representantene villige til å fortsette sine midlertidige roller i et
permanent styre.

Anders Folkestad ønsker å tre ut av styret, først og fremst fordi han ikke tilbringer så mye tid i
Oslo.

Interimsstyret vil foreslå at Andrii Bezkorovainyi velges til styremedlem. Han eier ikke leilighet,
men er leietaker, og har sagt seg illig til å være styremedlem. Dette foreslås fordi Andrii har
gitt oss et godt inntrykk og har kunskaper og ferdigheter som komplementerer de andre
styremedlemmene.

Vi har aldri hatt styremedlemer som ikke også har vært eiere før, men sameiemøtet har full
anledning til å velge et Andrii som styremedlem.

Det foreligger derfor følgende forslag til nytt styre:\bigskip

\begin{tabular}{@{\hspace{4em}}ll}
  {\bf Styrets leder:} & Knut Noremsaune
  \vspace{1ex}\\
  {\bf Styremedlemmer:} & Jenny Marie Myhre \\
  & Andrii Bezkorovainyi
  \vspace{1ex}\\
  {\bf Vararepresentanter:} & Pål Hermunn Johansen\\
  & Elisabeth Skoglund
\end{tabular}\medskip

Det skal stemmes over det nye styret, eller det kan foreslåss å gjøre valget ved akklamasjon, dersom
ingen har motforestillinger.

\end{document}
