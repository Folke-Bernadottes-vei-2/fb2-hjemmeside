% !TEX TS-program = LuaLaTeX
\documentclass[12pt]{article}

\usepackage[a4paper,margin=1in]{geometry}
\usepackage{enumitem}
\usepackage{hyperref}
\usepackage{libertine}
\usepackage[norsk]{babel}

\title{\LARGE\bf Innkalling til årsmøte}
\author{Årsmøte i Sameiet Folke Bernadottes vei 2}
\date{}

\setlength{\parindent}{0pt}
\setlength{\parskip}{1ex}
\setlength{\itemsep}{0.2ex}

\begin{document}

\begin{centering}
  {\LARGE\bf Innkalling til årsmøte}

  {\large Årsmøte i Sameiet Folke Bernadottes vei 2}

\bigskip

\begin{tabular}{ll}
  {\bf Dato og tid:} & Tirsdag 2. april 2024, klokken 18:00 \\
  & Møtet åpner for å ta imot møtedeltakere klokken 17:45
  \vspace{1ex}\\
  {\bf Sted:} & \href{https://us06web.zoom.us/j/87065838775?pwd=JbFFvkZyW09SXWT8135ZAohAycffxC.1}{Zoom-møte}\\
  & Møte-ID: 870 6583 8775 \\
  & Passord: 884841
\end{tabular}
\end{centering}

\bigskip

Neste side inneholder informasjon om møte-plattformen Zoom og hvordan man kan koble seg til.\bigskip

{\bf Til behandling foreligger:}
\begin{enumerate}[topsep=1pt,itemsep=1pt,partopsep=1pt, parsep=1pt]
\item[1.] Konstituering
  \begin{enumerate}[topsep=1pt,itemsep=1pt,partopsep=1pt, parsep=1pt]
    \item[a.] Registrering av deltakere
    \item[b.] Valg av møteleder
    \item[c.] Valg av protokollfører
    \item[d.] Valg av to sameiere til å underskrive protokollen
    \item[e.] Spørsmål om møtet er lovlig innkalt
  \end{enumerate}
\item[2.] Årsregnskap og revisors beretning for 2023 (eget vedlegg)
\item[3.] Gjennomgang av styrets arbeid
  \begin{enumerate}[topsep=1pt,itemsep=1pt,partopsep=1pt, parsep=1pt]
  \item[a.] Denne perioden
  \item[b.] Pågående saker
  \end{enumerate}
\item[4.] Innkomne saker
  \begin{enumerate}[topsep=1pt,itemsep=1pt,partopsep=1pt, parsep=1pt]
  \item[a.] Nytt ringeklokketablå (Styret)
  \item[b.] Fasade og balkonger (Elisabeth Skoglund)
  \item[c.] Tilrettelegge for salg av parkeringsplasser i FB4
  \item[d.] Forbedre avtaler med leverandører
  \item[e.] Øke fellesutgifter på grunn av økte kostnader (Pål Hermunn Johansen)
  \item[f.] Dreneringslag i bakgården (Maria Samuelsen)
  \item[g.] Endring av vedtekter angående dyrehold (Pål Hermunn Johansen)
  \item[h.] Flaggstang (Maria Samuelsen)
  \item[i.] Sanksjoner (Maria Samuelsen)
  \item[j.] Etablering av brannvarslingssystem og tilkobling til brannsentral for seksjoner i 2. etasje (Hans Jacob Hauge)
  \end{enumerate}
\item[5.] Budsjett
\item[6.] Valg av nytt styre
\end{enumerate}

\pagebreak
\section*{Angående digitalt avholdt møte}

Digitale årsmøter i sameier - og generalforsamlinger i borettslag - har blitt lovlig på permanent
basis. Styret ønsker at flest mulig skal kunne delta, og loven krever at det skal legges til rette
for at alle skal kunne delta. Styret har derfor valgt en møte-plattform som har god støtte av alle
de vanligste systemene for mobiltelefoner, nettbrett og datamaskiner. Zoom har apper/programmer for
Android, iOS, macOS, Linux og Windows, og det vil være mulig å bruke telefonen til å ringe inn og
delta (uten video).

For de som ikke kan delta på egne elektroniske enheter, vil det være mulig å gi fullmakt til en
annen sameier til å stemme på dems vegne. En slik fullmakt må meldes til styret, på ark,
tekstmelding eller e-post. Vi ønsker å ha slike fullmakter i hende dagen før møtet (eller
tidligere), slik at styret kan ha en best mulig oversikt når møtet starter.

En fullmakt til styret kan inkludere en forhåndsuttalelse og en instruks om hvordan det skal
stemmes i de ulike sakene. Forhåndsuttalelsene vil bli presentert på møtet, og kan også sendes på
forhånd til alle sameierne, hvis fullmaktsgiver ber om det.

Den beste måten å delta i møtet på er å følge lenken som ble delt sammen med e-posten med denne
innkallingen. Følg instruksjonene som dukker opp på skjermen for å komme inn i møtet.  Det vil være
mulig å komme inn i møterommet fra og med 15 minutter før møtet starter.

Vi håper at flest mulig vil delta i møtet via Zoom-applikasjonen, men ta kontakt med styret dersom
du vil delta via telefon, slik at vi kan teste det på forhånd. Vi ønsker ikke at eventuelle tekniske
problemer skal forhindre noen fra å delta i møtet

\pagebreak
\section*{2. Årsregnskap og revisors beretning}
Om budsjettet skriver regnskapsfører Jørn Sandberg følgende:

{\it Budsjettet for 2024 viser et overskudd på Kr. 315.893. Etter vår mening er dette et tilstrekkelig
overskudd for sameiet for 2024. Overskuddet skal dekke avdrag på lånet så dersom uforutsette
hendelser oppstår kan det bli behov for å justere felleskostnader.

Det var budsjettert et overskudd for 2023 på Kr. 541.614 men resultatet endte på Kr. 390.034. Noe
av årsaken til noe dårligere resultat skyldes den store renteøkningen som kom i 2023. Budsjetterte
renter var Kr. 245.167 men endte på Kr. 292.441. En endring på Kr. 47.274. Utover dette har
kommunale avgifter økt betydelig mer enn budsjettert. I tillegg har det vært noe mer vedlikehold på
bygg og heis enn budsjettert.}

For 2023 var utgiftene til heis i hovedsak knyttet til de gamle heisene, som vi hadde langt inn i
året som gikk. Det er ikke knyttet til de nye. Heiskostkadene vil gå betydelig ned nå som de er
byttet, og vi etablerer ny drifts- og fakturakontrolltavtale.

Sandberg skriver følgende angående heis for regnskapsåret 2023:

{\it Vedr. heis så har dere en vedlikeholdsavtale på som koster 10.000 i måneden. Det er allerede brukt Kr. 105.000 på heisen. Hvorav Kr. 18.870 egentlig skulle vært på fjoråret men kom i år siden de kom så sent. I tillegg har det kommet på Kr. 24.300 for Heisrådgiveren som heller ikke var i planene når budsjettet ble laget.

Heis:\\
Kr. 71.300 i januar 2023 for reparasjon dører som fusket.\\
2 x Kr. 51.993 for serviceavtale.\\
Kr. 8.437 til oslo kommune for sikkerhetskontroll

I tillegg div. kostnader Otis for heisstans, feil.}


\section*{3. Styrets arbeid}

\subsection*{3a. Styrets arbeid denne perioden}

Dette har vært en periode hvor det har blitt arbeidet videre med større saker fra foregående år, og styret ønsker å raskt gå gjennom hva som er gjort.

\begin{enumerate}
\item Ferdigstilling av heisene har vært et stort prosjekt, og det har opptatt mesteparten av tiden til Styret. Det har vært utfordringer med samarbeidet med Otis og Styret valgte å leie inn en uavhengig heiskonsulent for å forsikre oss om at vi får det resultatet vi har krav på. Han avdekket noen svakheter i arbeidet og derfor er ferdigstillingen fremdeles ikke ferdig, men det nærmer seg. Sameiet har holdt tilbake siste avdrag slik at heisene ikke blir betalt i sin helhet før dette er ordnet.
\item Branndørene til etasjene er byttet til moderne tette dører som faktisk har en funksjon ved brann. Det ble nevnt tidligere at de skulle få et magnetfeste for å stå åpne. Dette har vært en av konfliktene med Otis, og på dette punktet har vi ikke noe mer å gå på, dessverre. Men nytt Styre må gjerne få fortsette kampen om de mener det er hensiktsmessig bruk av ressurser.
\item Styret har inngått ny snømåkingsavtale med 1 hjelpende hånd. Dette ser vi på som en stor oppgradering fra tidligere år når det gjelder brøytingen. Firmaet har også mulighet til å bidra med annet vedlikehold, men vi har ikke avtalt noe mer pr. d.d.
\item Det har generelt vært en del uro i enkelte utleieleiligheter i bygget og Styret har brukt en del tid på dette, særlig på nyåret 2024 har det vært tidkrevende.
\item Det har på nytt vært gjort en nødvendig utbredning på taket og rundt heishuset, noe som heldigvis har gitt god effekt. I anledning årets snøsmelting har det ikke vært meldt om tette avløp eller lekkasjer, det er gledelig. Det har vært den største ugiftsposten i år, med tanke på at Styret har holdt tilbake på mye for å bygge opp økonomien igjen. Dette for å være helt sikker på at vi har økonomisk slingringsmonn dersom det skulle komme ytterligere rentehevinger, særlig med tanke på lånet vi har tatt opp. Derfor har ikke Styret brukt penger på vedlikehold som ikke har vært prekært. Det er flere ting som bør gjøres, der oppgradering av ringeklokketablå står øverst på vår liste. Vi har hentet inn anbud og vi anbefaler nytt styre til å prioritere dette om økonomien tillater det. Noe Sandberg har tatt høyde for i budsjettet for 2024. Ønsket Oppgradering/vasking av fasade, maling av trappehuset osv. er også noe vi har utsatt da vi har ønsket å ha mer penger i backup for å foreløpig unngå økning av felleskostnadene.
\item Styret har forsøkt å arrangere dugnad, men oppslutningen var lav. Det var tre som møtte opp, og to som ga svar at det ikke passet. Vi overlater med dette nytt Styre å forsøke å arrangere vårdugnad.
\end{enumerate}

Slik Styret ser det så er det mye bra som fungerer i Sameiet, men det er også ting som burde gjøres
for å øke trivselen. En visuell oppgradering av trappehuset, få malt dette, få vasket bygget
utvendig (særlig veggen mot Fb4) og ferdigstille uteområdet er punkter som er viktigere for et
bedre helhetsinntrykk.

Drenering bak blokken er fortsatt noe som burde gjøres, og El-billadere til uteplassen er satt på
pause, da vi ikke har prioritert dette økonomisk.

Styret har ikke tatt tak i prosjektet med takterasse i året som har gått, men dette er noe som
vekket interesse hos flere og som kan være spennende å se om nytt styre kan se videre på muligheten
for.

Man må være oppmerksom på at både vaskemaskinene og varmtvannstankene nærmer seg / har overskredet
sin forventede levetid, og derfor vil dette være utgifter som plutselig kan komme og som man må ta
høyde for i fremtidige prioriteringer. Da vil det nok innebære en økning av felleskostnadene for å
kunne dekke slike utgifter.

Styreleder Karlsen flyttet fra Fb2 i oktober 2023, samme gjorde Styremedlem Størmer. Hauge flytter
våren 2024 så det vil være helt avgjørende å samle et nytt Styre som melder seg til tjeneste på
årsmøtet 2.april og som er operativt fra samme dag.

Rent økonomisk vil det være praktisk å få noen inn i Styret som har mulighet til å kunne gjøre
enkelt vedlikehold selv for å slippe å leie inn folk for mindre oppgaver. Det hadde også vært lurt
med noen som har erfaring med jus, da Styret ofte må ta stilling til komplekse saker som krever mer
enn hva man kan forvente av en gjennomsnitlig beboer. Dette blir dyrt om man til en hver tid må
innhente hjelp utenfra.

Nåværende Styre takker for tilliten i året som har gått, og vi har forsøkt å gjøre vårt beste i et
år som vi ikke har hatt noe særlig økonomisk handlingsrom grunnet den økonomiske situasjonen i
samfunnet for øvrig, som også har påvirket oss i form av rentehevinger og dyrere tjenester.

\subsection*{3b. Pågående saker for styret}

På grunn av årets ekstraordinære situasjon der hele det sittende styret enten har solgt og flyttet
ut, eller har sin leilighet for salg, beskrives pågående saker her, for oppfølging av det neste
styret.

\begin{enumerate}
\item En luftavfukter for den nordligste delen av kjelleren, der de største bodene er samlet, skal
  få avløp. Vi venter på at rørleggeren skal gjøre sitt arbeid slik at luftavfukteren kan fungere uten manuell tømming.
\item Den tidligere løsningen for tørkerommet var brannfalig og kan ikke lenger brukes. Styret ønsker å
  vurdere forskjellige løsninger for å igjen kunne bruke rommet til tørking av klær.
\item Låsene på søppelinkastene er ikke egnet for utendørs bruk, og vi har en pågående sak med
  leverandøren av løsningen. Vi regner dette som en garantisak, og regner med å komme til en
  løsning uten for store kostnader for sameiet. Desverre er det tidvis veldig vanskelig å jobbe med
  slike saker, der en leverandør må utbedre feil uten mulighet for ekstra salg.
\item Branndørene i hver etasje (mellom gangen og trapperommet) er ikke levert slik de ble bestilt,
  og styret jobber med å få dette utbedret, men det er tidvis vanskelig å få leverandører til å
  prioritere utbedring av feil av denne typen.
\item Innkastet for den ene søppebrønnen har blitt bulket av et tilhengerfeste, og vi venter på
  pris for reparasjon fra leverandøren. Dessverre er det tidvis vanskelig å få raske svar på slike
  henvendelser.
\item Vi venter på et tilbud på service-avtale på søppelbrønnene, som deretter skal vurderes av
  styret. Dessverre har vi ikke fått et tilbud på dette ennå.
\item Styret jobber med å få flere tilbud for alternativ til dagens internett- og TV-løsning fra
  Telia, for eksempel fra GlobalConnect (eier av det tidligere Lynet, som har en tilstedeværelse i
  blokka).
\end{enumerate}

{\rule{\textwidth}{0.3pt}}

{\em Forslag til vedtak:}

Det avtredende styret skal overlevere detaljert informasjon om pågående saker presentert i
innkallingen, og det nye styret skal følge opp disse.\\
{\rule{\textwidth}{0.3pt}}


\section*{4. Inkomne saker}

I år er det flere innsendte saker en normalt, og derfor er det viktig at alle forbereder seg
godt. Når vi kommer til en sak på sakslisten, vil vi derfor direkte åpne for diskusjon rundt saken,
uten å først gå gjennom saksfremstillingen. Dersom ingen har kommentarer eller innspill, kan vi gå
rett til avstemming. Det er da ikke mulighet til å be om en pause for at møtedeltakerne skal få tid
til å lese teksten.

\subsection*{4a. Nytt ringeklokketablå}

{\em Forslagsstiller: Styret}

Styret anbefaler at vi inngår ny løsning for ringeklokketablået i stedet for å reparere på
eksisterende anlegg. Det er flere leiligheter som ikke har ringeklokke grunnet reseksjoneringer, og
det er flere som ikke har kontakt med sin ringeklokke uten at firmaet som vedlikeholder klarer å få
orden på dette. Styret har innhentet anbud og har en løsning fra Defigo som er fremtidsrettet og
økonomisk overkommelig.

Vedlagt er info om løsningen fra Defigo.

{\rule{\textwidth}{0.3pt}}

{\em Forslag til vedtak:}

Styret får tillatelse til å inngå en ny løsning for ringeklokketablået, enten med Defigo eller
tilsvarende løsning med annen leverandør. Styret kan inngå dette dersom dagens økonomiske situasjon
tillater det uten å gå på bekostning av felleskostnadene eller andre utgifter.\\
{\rule{\textwidth}{0.3pt}}

\subsection*{4b. Fasade og balkonger}

{\em Forslagsstiller: Elisabeth Skoglund}

Elisabeth Skoglund skriver i en mail til Styret med saker hun ønsker vi skal formidle på årsmøtet:

Gulv på veranda: Jeg har lagt merke til at gulvet på verandaen min har blitt skadet og trenger
reparasjon. Dette kan føre til farlige situasjoner om det ikke blir ordnet på riktig måte.

Maling av vinduer utvendig: Det trengs vedlikehold av vinduene for å unngå fuktighet og skade på
vindusrammene.

Påpeker også ikke fungerende dørtelefon i leiligheter, blant annet i 7045.

{\rule{\textwidth}{0.3pt}}

{\em Forslag til vedtak:}

Styret får oversikt over hva som trengs å gjøres med fasaden og innhenter anbud
for å se om det finnes økonomisk rom for dette, samt får en prioriteringsliste fra en fagmann der
vi blir oppdatert på hva som burde prioriteres og hva som kan vente.\\
{\rule{\textwidth}{0.3pt}}

\subsection*{4c. Tilrettelegge for salg av parkeringsplasser i FB4}

Utdrag fra epost til Styret fra en beboer:

FB2 eier 6 parkeringsplasser i FB4 som leies ut. Ved salg av én eller flere parkeringsplasser kan
deler av sameiets fellesgjeld betales ned. Salg av parkeringsplassene krever samtykke fra kommunen
og vedtektsendring i FB2 og FB4. Det er ønskelig at parkeringsplassene skal kunne selges til hvem
som helst. Hvis det ikke lar seg gjøre er det ønskelig at parkeringsplassene skal kunne selges til
beboere i Folke Bernadottes vei, eller eventuelt beboere i FB4.

{\rule{\textwidth}{0.3pt}}

{\em Forslag til vedtak:}

Styret går i dialog med kommunen med mål om å få samtykke til mulige salg av parkeringsplasser FB2
eier i FB4. Etter å ha fått samtykke fra kommunen jobber styret opp mot styret i FB4 for å komme
til en løsning som gjør at FB2 har mulighet til å selge
parkeringsplassene.\\
{\rule{\textwidth}{0.3pt}}

\subsection*{4d. Forbedre avtaler med leverandører}

Utdrag fra epost til Styret fra en beboer:

Det er viktig at styret følger opp avtaler med leverandører for å kutte kostnader, forhindre at
kostnader øker mer enn strengt nødvendig og sikre at sameiet mottar varer og tjenester med høy
kvalitet.

{\em Forslag til vedtak:}

Styret etablerer en rutine for å gjennomgå og vurdere kostnader og leveranser knyttet til
betydelige leverandører (inkludert heisleverandør) hvert år, herunder undersøker muligheten for å
flytte leveranser (for eksempel strøm, forsikring, lån, TV, internett og renhold) til andre
leverandører ved å innhente flere tilbud og forhandle med eksisterende leverandører. Første
gjennomgang gjennomføres i 2024.\\
{\rule{\textwidth}{0.3pt}}

\subsection*{4e. Øke fellesutgifter på grunn av økte kostnader}

{\em Forslagsstiller: Pål Hermunn Johansen}

Fellesutgiftene har ikke endret seg siden 1. januar 2023 samtidig som prisstigningen har vært over
4\% per år. Det betyr at Sameiets kjøpekraft har gått kraftig ned det siste året. For å kompansere
for dette bør vi øke fellesutgftene, først fra 1. juli 2024, og deretter fra 1. januar 2025. Merk
at forslaget under ikke vil gi Sameiet nevneverdig økt kjøpekraft sammenliknet med nivået fra 2023,
men vil gi styret økonomisk handlerom for å utføre nødvendig vedlikehold og betale uventede
utgifter uten å måtte kalle inn til et ekstraordinært sameiemøte.

Regnskapsføreren vår, Jørn Sandberg, bemerker at vi strengt tatt ikke trenger å øke
fellesutgiftene, men jeg mener at vi likevel bør gjøre det. Det er sannsynlig at utgifter vil øke
på tross av at prisstigningen nå ser ut til å gå ned, og det er også sannsynlig at bygningsmassen
har behov for utbedringer i nær fremtid.

Forslaget kan og bør stemmes over i saken som omhandler budsjettet, etter at alle de innkomne
sakene er behandlet.

{\rule{\textwidth}{0.3pt}}

{\em Forslag til vedtak:}
\begin{enumerate}
\item[i.]Fra 1.\ juli økes fellesutgiftene økes til 76 kroner per andel i sameiebrøken. Dette er en
  økning på om lag 4.5\% fra dagens 72,75 per andel, og dette vil bety at de minste leilighetene
  vil måtte betale 2128 kroner per måned.
\item[ii.]Fra 1.\ januar økes fellesutgiftene økes til 80 kroner per andel i sameiebrøken. Dette er
  en økning på om lag 5\% fra høstens nivå på 76 kroner per andel, som vil bety at de minste
  leilighetene vil måtte betale 2240 kroner per andel.
\end{enumerate}
{\rule{\textwidth}{0.3pt}}

Subsidiært foreslås det samme med henholdsvis 75 kroner per andel fra 1.\ juli og 78 kroner per
andel fra 1.\ januar. Det vil si, dersom forslaget over ikke vedtas, bør det samme forslaget
stemmes over med reduserte valører.

\subsection*{4f. Dreneringslaget i bakgården}

{\em Forslagsstiller: Maria Samuelsen}

Som de fleste i blokka sikkert har fått med seg, ble kjelleren i blokka fylt med flytende kumøkk fra
Folke Bernadottes vei 4 (FB4) en tid tilbake under byggingen av FB4. Utbyggerne,
PEAB/Naturbetong, har nektet å betale for utbedring av skadene og vask av kjeller. FB2 hadde
heldigvis innboforsikring da det skjedde, men PEAB har ikke engang villet betale egenandelen for
vask av kjelleren.

Flere av beboerne i blokka er handikappede og uføre, og mange beboere måtte kaste mye eller alt de
hadde i kjellerbodene. Mange ting var uerstattelige, og flere av beboerne har ikke råd til
innboforsikring. Slik jeg har forstått forurensningsloven, er man pliktig til å rydde opp forurensning
som man forårsaker, uansett om man har skyld i det eller ikke, men PEAB, et firma med en
omsetning på 58 milliarder kroner i 2021, har ikke engang vært villig til å betale 9 280,- kr. i
egenandel til forsikringsselskapet til sameiet for vask av kjelleren. De vil heller ikke betale for
utskifting av tett dreneringslag. PEAB fremstår som en klassisk skurk i en barnefilm, bare at dette er
virkeligheten. Kontrasten til hjemmesiden deres blir slående: \href{https://peab.no/om-peab/kjerneverdier/}{https://peab.no/om-peab/kjerneverdier/}

Jeg har brukt mye tid og krefter på å kartlegge omstendighetene rundt saken og krangle med PEAB,
ikke bare for å få PEAB til å betale for utskifting av dreneringslaget og egenandelen for kjellervask,
men også fordi overvannet fra FB4 fortsatt renner ned i bakgården vår, der det legger igjen
sedimenter. Dreneringslaget må derfor skiftes ut hyppigere etter utbyggingen av FB4, enn før
utbyggingen. Den tekniske løsningen for Folke Bernadottes vei 4 påfører derfor Sameiet Folke
Bernadottes vei 2 store løpende kostnader.

\href{https://www.bondelaget.no/jusilandbruket/foring-av-overflate-og-dreneringsvann-inn-pa-naboens-grunn}{https://www.bondelaget.no/jusilandbruket/foring-av-overflate-og-dreneringsvann-inn-pa-\\naboens-grunn}

{\em{}[...jfr. grannelova § 9. Vilkåret for erstatningskrav, er at naboen – i dette tilfellet A «… har, gjør
eller iverksetter noe som er til urimelig eller unødvendig skade eller ulempe på naboeiendom», jf
grannelova § 2. Når overflate- eller drensvann med hensikt ledes inn på naboens eiendom, rammes
man klart av forbudet i § 2. Men § 2 kan i noen tilfeller også ramme mindre graverende tilfeller, for
eksempel når dreneringen er dårlig og store mengder vann stadig kommer over på naboen.]}

Jeg har snakket med en ansatt i kommunen som ba meg om å henvende meg skriftlig til kommunen
og spørre om det hadde skjedd noen ulovligheter ved byggingen. Svaret var overraskende. Det viste
seg at kommunen har godkjent byggingen. De vil derfor selvfølgelig ikke innrømme å ha gjort noe
feil, da de kan komme i en kostbar konflikt med PEAB. Jeg opplever fremferden fra kommunen
som ganske snikete i denne saken, da de mest sentrale dokumentene er skjult bak et krav om digital
innlogging. Jeg sitter med inntrykket av at de har kalkulert med at ingen andre enn en utbrent
styreleder, travelt opptatt med å krangle med PEAB og Naturbetong med et høyt konfliktnivå over
lang tid, har hatt muligheten til å se over nabovarselet. Så vidt jeg vet, var ingen i styret
anleggsingeniører i perioden da konflikten foregikk, og da er det heller ikke gitt at man forstår
konsekvensene av en radikal terrengheving på nabotomta.

Situasjonen i dag er at dreneringslaget bak blokka er helt tett av kumøkk. Dette ødelegger
grunnmuren, og Sameiet Folke Bernadottes vei 2 har ikke penger til å betale for utskifting av
dreneringslaget selv. Dessverre har progresjonen i denne saken gått saktere i perioder, da jeg selv
har vært svært utbrent.

Dreneringslaget bak blokka fungerte svært godt før utbyggingen av FB4.

{\rule{\textwidth}{0.3pt}}

{\em Forslag til vedtak:}

Styret skaffer en vurdering av saken hos et dyktig advokatfirma for å se på muligheten til
\begin{itemize}
\item[-] å få PEAB til å betale for utskifting av dreneringslag, vask av kjelleren og
saksomkostningene
\item[-] å vurdere lovligheten i konstruksjonen med hensyn på å sende overvannet ned i bakgården
vår, med formål å finne en bedre løsning som ikke sender overvannet fra FB4 ned i bakgården
til FB2.
\end{itemize}
{\rule{\textwidth}{0.3pt}}

Vedlegg:
Svar fra Plan- og bygningsetaten i Oslo kommune - 1\_Forespørsel besvart

Henvisning:

Folke Bernadottes vei 4 - Oppføring av boligblokk - Tidligere adresse Folke Bernadottes vei 2,
byggesak 201705305, dokumentnummer 90, fil 1 - 9:

\href{https://innsyn.pbe.oslo.kommune.no/saksinnsyn/docdet.asp?jnr=2018115290\&sti=fromResult\&caseno=201705305}{https://innsyn.pbe.oslo.kommune.no/saksinnsyn/docdet.asp?\\jnr=2018115290\&sti=fromResult\&caseno=201705305}

Folke Bernadottes vei 4 - Oppføring av boligblokk - Tidligere adresse Folke Bernadottes vei 2,
byggesak 201705305, dokumentnummer 131, fil 1 - 4:

\href{https://innsyn.pbe.oslo.kommune.no/saksinnsyn/docdet.asp?jnr=2019054092\&sti=fromResult\&caseno=201705305}{https://innsyn.pbe.oslo.kommune.no/saksinnsyn/docdet.asp?\\jnr=2019054092\&sti=fromResult\&caseno=201705305}

Merk at ``Drensledning'' på Overvannsplanen ligger rundt det nedre garasjeplanet og ikke like under plenen.

\subsection*{4g. Endring av vedtekter angående dyrehold}

{\em Forslagsstiller: Pål Hermunn Johansen}

I husordensreglene er følgende punkter (se \href{http://fb2.no/nyttig/dyrehold/}{Sameiets regler om dyrehold}):

\begin{itemize}
\item Lufting av hund/katt skal skje utenom at det er til sjenanse for andre.
\item Lufting av hund må skje under kontroll. Ekskrementer skal fjernes uansett om det er på sameiets område eller ikke.
\end{itemize}

Det hender at hunder som bor i blokka, tisser på og/eller svært nær blokka og på innkastene til dypoppsamlerne. Dette er en uting! Når en hund begynner å tisse et sted, blir det et etablert hundetoilett på stedet, og problemet eskalerer.

{\rule{\textwidth}{0.3pt}}

{\em Forslag til vedtak:}
\begin{enumerate}
\item[1.] Husordensreglene som gjelder dyrehold endres ved at det legges til følgende punkt, under punktene angående lufting av hund og katt:
\begin{itemize}
\item Ved lufting av hund skal eieren se til at hunden ikke markerer med urin på husets vegger, ved husets inngangsparti eller ved/på husets søppelanlegg. Overtredelse av denne regelen regnes som et grovt regelbrudd av husordensreglene.
\end{itemize}
\item[2.] Styret skal ha oppslag angående regler for husdyrhold i Sameiets fellesareal.
\end{enumerate}
{\rule{\textwidth}{0.3pt}}

\subsection*{4h. Flaggstang}

{\em Forslagsstiller: Maria Samuelsen}

Da FB4 skulle bygges, måtte flaggstangen til FB2 fjernes. Den stod tidligere på en bergknaus der
innkjørselen til FB4 nå ligger. Etter det, har den ligget lagret bak blokka. Jeg tenker at nederst på
parkeringsplassen er det eneste stedet på tomta vår der den ikke kommer i konflikt med parkering
og brøyting. Dessverre er dette også mindre enn 15 meter fra kommunal vei. Flaggstangens tidligere
plassering var også innenfor denne grensen, men jeg antar at den ble satt opp uten å søke om det
først.

{\rule{\textwidth}{0.3pt}}

{\em Forslag til vedtak:}

Styret søker kommunen om å få sette opp flaggstangen lengst sør på parkeringsplassen rett
nord for det nye rosebedet.\\
{\rule{\textwidth}{0.3pt}}

Alternativt bør den selges.

\subsection*{4i. Sanksjoner}

{\em Forslagsstiller: Maria Samuelsen}

Den eneste sanksjonen styret har overfor eiere som bryter reglene, er tvangssalg. Jeg skjønner at
det sitter langt inne å gå så langt, så jeg etterlyser andre sanksjonsformer for eiere som bryter
husreglene eller ignorerer pålegg fra styret om byggetekniske utbedringer.

Et eksempel er en eier som har montert mekaniske avtrekksvifter på ventilasjonsanlegget på kjøkken
og bad. Dette er søknadspliktig overfor kommunen og ikke tillatt i denne blokka. Årsaken til
regelen er brannfare, men regelbrudd byr på flere problemer enn det.

Resultatet for vårt vedkommende, var at vi fikk mat-osen fra denne leiligheten, med voldsom kraft,
inn i vår leilighet. Dette er selvfølgelig svært påtrengende, inngripende og plagsomt. De som vi
kjøpte leiligheten av, hadde hatt det samme problemet, men fant ikke ut hvor det kom fra. Det tok
lang tid før vi fikk adressert problemet fordi eieren i mange år nektet for å ha montert propell på
ventilasjonsåpningene. Tilfeldigheter gjorde at dette likevel ble avslørt. Eieren koblet tidligere
fra de mekaniske viftene før en gjennomgang av hele ventilasjonssystemet og monterte dem deretter
på igjen, vel vitende om at dette var til plage for andre beboere. Jeg skal ikke bruke ordet
«drittsekk» i denne sammenhengen, selv om det hadde vært helt passende.

Etter at problemet var lokalisert, nektet fortsatt eieren å demontere viftene, men ba leietakerne
om å la være å bruke kjøkkenviften. Problemet gjentok seg selvsagt for hvert nye leieforhold, og
siden propellen fortsatt var i drift på badet, var vi egentlig like langt.

Jeg har ikke tall på hvor mange klager jeg har sendt til styret om dette problemet, men etter at
jeg begynte å nummerere dem, har jeg kommet til 6. Tvangssalg burde ha vært på bordet for lenge
siden.

{\rule{\textwidth}{0.3pt}}

{\em Forslag til vedtak:}

Styret skal utarbeide ytterligere sanksjonsformer, som for eksempel døgnmulgt, for eiere som
nekter å rette seg etter husreglene eller utføre pålagte utbedringer.\\
{\rule{\textwidth}{0.3pt}}

\subsection*{4j. Etablering av brannvarslingssystem og tilkobling til brannsentral for seksjoner i 2. etasje}

{\em Forslagsstiller: Hans Jacob Hauge}

Det er i dag minst tre av de fire leilighetene som ble reseksjonert i 2022 i 2. etg som i dag ikke
er tilkoblet felles brannsentral i bygget. For å øke brannsikkerheten i bygget anser jeg det som
svært viktig at de resterende leilighetene også får etablert tilsvarende detektorer som øvrige
beboere, og som kobles til brannsentralen. Jeg har foreløpig ikke noe tilbud eller kostnadsestimat
på disse arbeidene, men det er en såpass viktig sak at det bør prioriteres. Siden dette er et
tiltak som påvirker alle – og ikke bare eierne av de enkelte leilighetene, foreslås det at Sameiet
bekoster tiltaket.

{\rule{\textwidth}{0.3pt}}

{\em Forslag til tiltak:}

Sameiet bekoster detektorer, tilkoblingsarbeider og eventuelt andre direkte kostnader ifm tiltaket, og avtaler videre med eierne av leilighetene i 2. etg for etablering av dette.\\
{\rule{\textwidth}{0.3pt}}

\section*{5. Budsjett}

Forslag til budsjett finnes på samme vedlegg som regnskapet, men diskuteres etter andre saker.

\section*{6. Valg av nytt styre}

Per dags dato (utsendelse av denne innkallingen), er valgkomitéen bare delvis i havn med å finne
kandidater til neste års styre. Den tentative listen er som følger:

\begin{tabular}{ll}
  Styrets leder: & Jenny Marie Myhre
  \vspace{1ex}\\
  Styremedlemmer: & Jenny Marie Myhre
  \vspace{1ex}\\
  Vararepresentanter: & Pål Hermunn Johansen\\
  & Elisabeth Skoglund
\end{tabular}

Dersom valgkommitéen kan, vil en mer fullstendig liste presenteres på møtet.

\end{document}
